\documentclass[12pt]{article} % Default font size is 12pt, it can be changed here

\usepackage{geometry} % Required to change the page size to A4
\geometry{a4paper} % Set the page size to be A4 as opposed to the default US Letter

\usepackage{graphicx} % Required for including pictures

\usepackage{float} % Allows putting an [H] in \begin{figure} to specify the exact location of the figure
\usepackage{wrapfig} % Allows in-line images such as the example fish picture

\usepackage{lipsum} % Used for inserting dummy 'Lorem ipsum' text into the template

\linespread{1.2} % Line spacing

%\setlength\parindent{0pt} % Uncomment to remove all indentation from paragraphs

\graphicspath{{Pictures/}} % Specifies the directory where pictures are stored

\begin{document}

%----------------------------------------------------------------------------------------
%	TITLE PAGE
%----------------------------------------------------------------------------------------

\begin{titlepage}

\newcommand{\HRule}{\rule{\linewidth}{0.5mm}} % Defines a new command for the horizontal lines, change thickness here

\center % Center everything on the page

\textsc{\LARGE Siena College}\\[1.5cm] % Name of your university/college
\textsc{\large 1.00 Credit Independent Study Proposal}\\[0.5cm] % Minor heading such as course title
\HRule \\[0.5cm]
{ \huge \bfseries --------- Particle Physics  --------- Exploring Alternate Interpretations of the Pentaquark Candidate Observed by LHCb}\\[0.4cm] % Title of your document
\HRule \\[1.5cm]

\begin{minipage}{0.4\textwidth}
\begin{flushleft} \large
\emph{Author:}\\
\textsc{Sandy Spicer} % Your name
\end{flushleft}
\end{minipage}
~
\begin{minipage}{0.4\textwidth}
\begin{flushright} \large
\emph{Supervisor:} \\
\textsc{Dr. Bellis} % Supervisor's Name
\end{flushright}
\end{minipage}\\[4cm]

{\large \today}\\[3cm] % Date, change the \today to a set date if you want to be precise

\vfill % Fill the rest of the page with whitespace

\end{titlepage}

%----------------------------------------------------------------------------------------
%	INTRODUCTION
%----------------------------------------------------------------------------------------

\section{Proposal} % Major section

%------------------------------------------------

\subsection{Rationale for Conducting an Independent Study} % Sub-section

I am proposing to engage in this Independent Study because I would like to learn the basics of particle physics as well as gain more experience with Python programming.  In this Independent Study, I will strengthen my Python programming and critical thinking skills. This research experience will prepare me for future course work and grad school, if I choose to pursue that path. In the future, I hope to pursue research in Particle Physics and would like to start gaining experience right away. 

Python is an essential language in the field of physics and this Independent Study will help me master it. I have learned the basics of Python this year in my CSIS 200 course, and I hope to strengthen these skills so that I am able to code on my own. Additionally, this research experience will introduce me to all of the fundamental particles, including quarks, leptons, and bosons. I will learn what particles are classified under these categories, their properties, and the role they play in the universe.

%------------------------------------------------

\subsection{Plan of Action} % Sub-section

The purpose of this Independent Study is to explore alternate interpretations of the pentaquark observed by LHCb: a particle built of five quarks. Many physicists believe experimental results point toward the discovery of a new particle, known as the pentaquark. However, it is always good to check the work of others. Data from the Department of Energy's Jefferson Lab reveals that the one pentaquark candidate did not hold up to additional experimental scrutiny. 

In order to prepare for this analysis, I have read through {\it Particle Physics: A Graphic Guide} by Tom Whyntie and Oliver Pugh, which has given me an introduction to the field of Particle Physics. In addition, I have started working through particle physics tutorials on the Particle Physics Playground website. These tutorials are available online and use actual data from CERN. In the activities that I have done so far, I have been able to use coding to determine masses of particles that arise from collisions. Using the calculated masses, I have been able to plot histograms to visualize and identify the particles. These introductory activities have helped me form the fundamental understanding needed to analyze data on my own. 

I will conduct the analysis by using a similar approach to the Particle Physics Playground activities. Since one can never be entirely sure of what particles arise from collisions, it is easy to make mistakes in assigning particles. After I come to a conclusion, Dr. Bellis will introduce me to some of his current research in Particle Physics.

%------------------------------------------------

\subsection{Learning Goals/Objectives} % Sub-section

My learning goals for this Independent Study are as follows:
\begin{itemize}
  \item Does the Pentaquark exist?
  \item Develop a working knowledge of relativistic kinematics
  \item Understand why you cannot rely on classical physics to determine the momentum, mass, and velocity of a particle in Particle Physics
  \item Be able to identify what type of particles arise from collisions, their masses, and what they decay to
  \item Be able to use Python to calculate the masses and plot histograms of data
  \item Learn the masses of particles as well as what they commonly decay to
\end{itemize}

%------------------------------------------------

\subsection{Expected Project Delivery Date} % Sub-section

The expected project delivery date for this Independent Study is May 2016.

%----------------------------------------------------------------------------------------

\subsection{Bibliography/Required Reading} % Sub-section

{\it Particle Physics: A Graphic Guide} by Tom Whyntie and Oliver Pugh


\end{document}